% CUHKexam.tex (Version 1.0)
% ===============================================================================
% The Chinese University of Hong Kong Exam LaTeX template.
% 2024, Yuxi Chen, CUHK Dept. of Psychology
% Modified based on the 2004; 2009, Timothy Kam, ANU School of Economics ANU Exam template
% Licence type: Free as defined in the GNU General Public Licence: http://www.gnu.org/licenses/gpl.html

\documentclass[a4paper,12pt,fleqn]{article}
\usepackage{amsmath}
\usepackage{enumerate}
\usepackage{enumitem}
\usepackage{fancyhdr}
\usepackage{graphicx}

% Insert your course information here %%%%%%%%%%%%%%%%%%%%%%%%%%%%%%%%%%

\newcommand{\institution}{THE CHINESE UNIVERSITY OF HONG KONG}
\newcommand{\titlehd}{COURSE TITLE}
\newcommand{\examtype}{Exam Type}
\newcommand{\examdate}{Exam Date}
\newcommand{\examcode}{Course Code}
\newcommand{\version}{A}
\newcommand{\readtime}{15 Minutes}
\newcommand{\writetime}{45 Minutes}
\newcommand{\questionnumber}{4}
\newcommand{\materials}{Non-programmable Calculators}
\newcommand{\lastwords}{End of Examination}

%%%%%%%%%%%%%%%%%%%%%%%%%%%%%%%%%%%%%%%%%%%%%%%%%%%%

%\setcounter{MaxMatrixCols}{10}
\newtheorem{theorem}{Theorem}
\newtheorem{acknowledgement}[theorem]{Acknowledgement}
\newtheorem{algorithm}[theorem]{Algorithm}
\newtheorem{axiom}[theorem]{Axiom}
\newtheorem{case}[theorem]{Case}
\newtheorem{claim}[theorem]{Claim}
\newtheorem{conclusion}[theorem]{Conclusion}
\newtheorem{condition}[theorem]{Condition}
\newtheorem{conjecture}[theorem]{Conjecture}
\newtheorem{corollary}[theorem]{Corollary}
\newtheorem{criterion}[theorem]{Criterion}
\newtheorem{definition}[theorem]{Definition}
\newtheorem{example}[theorem]{Example}
\newtheorem{exercise}[theorem]{Exercise}
\newtheorem{lemma}[theorem]{Lemma}
\newtheorem{notation}[theorem]{Notation}
\newtheorem{problem}[theorem]{Problem}
\newtheorem{proposition}[theorem]{Proposition}
\newtheorem{remark}[theorem]{Remark}
\newtheorem{solution}[theorem]{Solution}
\newtheorem{summary}[theorem]{Summary}
\newenvironment{proof}[1][Proof]{\noindent\textbf{#1.} }{\ \rule{0.5em}{0.5em}}

% Margins and footer style
\setlength{\topmargin}{0cm}
\setlength{\textheight}{9.25in}
\setlength{\oddsidemargin}{0.0in}
\setlength{\evensidemargin}{0.0in}
\setlength{\textwidth}{16cm}
\pagestyle{fancy}
\lhead{} 
\chead{} 
\rhead{} 
\lfoot{} 
\cfoot{\footnotesize{Page \thepage \ of \pageref{finalpage} -- \titlehd \ (\examcode)}} 
\rfoot{} 


\renewcommand{\headrulewidth}{0pt} %Do not print a rule below the header
\renewcommand{\footrulewidth}{0pt}


\begin{document}

% Title page

% Logo
\begin{center}
    \includegraphics[width=4cm]{Emblem_of_CU.png}
\end{center}

\vspace{0.5cm}

\begin{center}
\Large\textbf{\institution}
\end{center}
%\vspace{0.25cm}

\begin{center}
\textit{ \examtype -- \examdate}
\end{center}
\vspace{0.25cm}

\begin{center}
\large\textbf{\titlehd}
\end{center}

\begin{center}
\textbf{\examcode}
\end{center}



\begin{center}
\large\textbf{Version: \version}
\end{center}
\vspace{0.25cm}
\begin{center}
\textbf{Number of Questions: \questionnumber}
\end{center}
\begin{center}
\textbf{Writing Time:  \writetime}
\end{center}
\begin{center}
\textbf{Permitted Materials: \materials}
\end{center}
\vspace{0.5cm}

\noindent Name: \underline{\hspace{5cm}} \hspace{2cm} Student ID: \underline{\hspace{5cm}} \hspace{0.5cm} 
\vspace{0.5cm}

\noindent\underline{\bf INSTRUCTIONS}
\vspace{0.25cm}

% Instruction

\begin{enumerate}
\item This quiz contains {\bf 5} questions and comprises 
{\bf 5} printed pages.
\item Please write down your answers on the answer sheet. 
\item Please write down your version number on the answer sheet.
\item This is a {\bf CLOSE BOOK} quiz.
\item {\bf Return BOTH sheets (question and answer sheets) to the tutor at the end of the quiz}
\end{enumerate}

% End title page

% Question page

\newpage
% Question type title
\paragraph{\textbf{\underline{Question Type Title}}}

~\\~\\ \textit{Write your specification of this type of questions here.}

% Multiple-choice questions
\bigskip
\paragraph{\textbf{Question 1}}
This is an example of a multiple-choice question. For blank in question use \underline{\hspace{2cm}}.  
\begin{enumerate}[label=\alph*.]
    \item option 1
    \item option 2
    \item option 3
    \item option 4
\end{enumerate}

% Fill-in-blank questions
\newpage
\paragraph{\textbf{Question 2}}
This is an example of a fill-in-blank question
\begin{enumerate}
    \item The process of photosynthesis occurs in \underline{\hspace{5cm}}.
    \item Water freezes at \underline{\hspace{2cm}} degrees Celsius.
\end{enumerate}

% True-or-False questions
\newpage
\paragraph{\textbf{Question 3}}
This is an example of a True or False question. Mark the following True or False
\begin{enumerate}
    \item \underline{\hspace{0.5cm}} The Earth revolves around the Sun.
    \item \underline{\hspace{0.5cm}} Water boils at 100 degrees Celsius at standard atmospheric pressure.
    \item \underline{\hspace{0.5cm}} The capital of Australia is Sydney.
    \item \underline{\hspace{0.5cm}} All cats are black.
\end{enumerate}

% Short answer questions
\newpage
\paragraph{\textbf{Question 4}}\hfill (12 marks)\\
This is an example of a short answer question with sub-questions.
\begin{enumerate}
    \item Part 1 of question. \hfill (2 marks)
    \item Part 2 of question. \hfill (4 marks)
    \item Part 3 of question. \hfill (6 marks)
\end{enumerate}



\begin{center}
\vspace{3cm}
--------- \textit{\lastwords} ---------
\end{center}


\label{finalpage}

\end{document}
